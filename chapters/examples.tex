\chapter{Anwendungsbeispiel}

\section{Schlüsselerzeugung}

Für die Erzeugung eines Schlüsselpaares werden zwei zufällige große Primzahlen $p$ und $q$ gewählt.
Um das Errechnen von $p$ und $q$ durch Faktorisierung von $n$ zu Erschweren sollten Primzahlen für $p$ und $q$ gewählt werden, welche beide \textquote{groß}\footnote{Derzeitig sichere RSA-Schlüssel besitzen zwischen $1024$-und $4096$-Bits beziehungsweise $128$-und $512$-Bytes} und \textquote{weit} auseinander liegen. Beide Primfaktoren $p$ und $q$ sind durch die Bitgröße von $n$, für welche in der Regel eine Zweierpotenz gewählt wird, beschränkt, wobei $p$ und $q$ normalerweise entweder gleich groß oder annähernd gleich groß sind. Für jeden Wert $n$ ist die Bytegröße $nB$:
\begin{align}
  \{nB \in \mathbb{N}\;|\; 6 \le nB \le 512 \}\\
  nB=\lfloor\log_2{n}\rfloor+1=\lfloor\frac{\log{n}}{\log{2}}\rfloor+1
\end{align}
So ergibt sich für $nB=512$ $p$, $q$ und $n$:
\begin{align}
\{p \in \mathbb{N}\;|\; 5 < p \le 2^{256}-1 \}\\
\{q \in \mathbb{N}\;|\; 11 < q \le 2^{256}-1 \}\\
\{n \in \mathbb{N}\;|\; 55 \le n \le 2^{512}-1 \}
\end{align}

\newpage
\section{Schlüsselverteilung}
\newpage
\section{Verschlüsselung}
\newpage
\section{Entschlüsselung}
\newpage
\section{Signatur}
