\section{Schlüsselerzeugung}

Die Erzeugung eines Schlüsselpaares basiert auf der Selektion zweier kryptographisch sicheren zufälligen Primzahlen, standardmäßig als $p$ und $q$ bezeichnet, welche meistens beide die gleiche Bitgröße besitzen, dies ist jedoch nicht zwingend notwendig. Die größte Herausforderung stellt das effiziente\footnote{Normalerweise in 1-20 Sekunden} Finden dieser großen Primzahlen \char"2012{} 1024-bit, 2048-bit, \ldots{} \char"2012{} dar. Die von mir verwendete Selektion der Primzahlen basiert auf diesem grundlegenden Prozess:

\begin{enumerate}
  \item Zufällige Zahl mit gegebener Bitgröße generieren
  \item Variable \textquote{failed} mit Standardwert false deklarieren
  \item Modulo der Generierten Zahl und der 5000 ersten Primzahlen vergleichen
    \begin{enumerate}
      \item Falls Modulo gleich null
      \item p um zwei erhöhen
      \item failed auf true setzen
      \item Zurück zu Schritt 2 gehen
    \end{enumerate}
  \item Fermat-Test durchführen
    \begin{enumerate}
      \item Falls fehlgeschlagen
      \item Zurück zu Schritt 1 gehen
    \end{enumerate}
  \item Miller-Rabin-Test durchführen
    \begin{enumerate}
      \item Falls fehlgeschlagen
      \item
    \end{enumerate}
\end{enumerate}

\begin{figure}[H]
  \begin{cminted}

  // BigInteger ist ein Datentyp welcher beliebig große Ganzzahlen darstellen kann
  BigInteger p = 0;

  // p und q werden gleichzeitig berechnet, p als t1 und q als t2 (hier ausgelassen)
  var t1 = new Task(() =>
  {
    // Dies ist ein Label zu welchem jederzeit zurückgegangen werden kann
    Restart:

    // Diese Funktion setzt p auf eine kryptographisch sichere 2048-bit/2056-byte BigInteger-Zahl.
    p = GetRandomBigInt(256);

    // Setze das LSB_0-bit auf 1, da alle Primzahlen größer 2 ungerade sind
    p |= 1 << 0;

    Failed:

    bool failed = false;
  \end{cminted}
  \caption{Erste Schritte der Primzahlauswahl}
\end{figure}
\begin{figure}
  \begin{cminted}

    // Parallelisierte Probedivision der zu untersuchenden Zahl durch die ersten 5000 Primzahlen
    Parallel.ForEach(Primes.List, (prime, state) =>
    {
      // Falls p mod prime gleich Null ist ist p eine zusammengesetzte Zahl
      if (p % prime == 0)
      {
        // Wenn p zusammengesetzt ist um zwei erhöhen, die failed-Flagge setzen und aus der parallen Berechnung ausbrechen
        p += 2;
        failed = true;
        state.Break();
      }
    });

    // Falls Probedivision fehlschlug zurück zu Failed gehen und mit dem um zwei erhöhten p erneut versuchen
    if (failed)
      goto Failed;
  \end{cminted}
  \caption{Zweiter Schritt der Primzahlauswahl}
\end{figure}
\begin{figure}
  \begin{cminted}

    // Eine Iteration des Fermat-Testes ausführen und falls dieser fehlschlägt mit einem komplett neuen p beginnen
    if (!Fermat(p, 1))
      goto Restart;

    // 5 Iterationen des Miller-Rabin-Testes ausführen und falls dieser p als Primzahl einstuft die Ausführung der Task beenden
    if (MillerRabin.IsPrime(p, 5))
      return;
            
    // Falls Miller-Rabin fehlschlägt wieder komplett neu starten
    goto Restart;
  \end{cminted}
  \caption{Letzte Schritte der Primzahlauswahl}
\end{figure}
