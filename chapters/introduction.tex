\chapter{Einleitung}

Kryptographie und Verschlüsselungssysteme wurden schon im Antiken Rom verwendet um Nachrichten zu verschlüsseln\cite{aichner22}, doch heutzutage wären Verschlüsselungssysteme wie der \textquote{Cäsar-Chiffre} gegenüber eines Computers nutzlos.
Die Rolle von Kryptographie ist aufgrund des leichten Zugangs der Allgemeinheit zu leistungsstarken Computern, welche das Brechen von \textquote{schwachen} Verschlüsselungssystemen automatisiert und beschleunigt haben, um einiges angestiegen.
Aufgrund der nun verfügbaren Rechenleistung wurde die Notwendigkeit für stärkere Verschlüsselungssysteme immer größer. Zur Zeit verwendete Verschlüsselungssysteme sind so konzipiert, dass sie mathematisch schlecht rückwärts zu berechnen sind.
Hierzu werden standardmäßig große Primzahlen verwendet, da zur Zeit kein effizienter Algorithmus zur Berechnung von großen Primzahlfaktoren bekannt ist, welcher nicht die Verwendung eines Quantencomputers erfordern würde.
Das \textquote{brute-forcing} moderner Verschlüsselungssysteme ist so ineffizient, dass nur ein Teilschritt bis zu fünfzehnhundert Jahre dauern kann\cite{kleinjung10}.
\textquote{Für unsere Berechnungen waren mehr als $10^{20}$ Operationen erforderlich. Mit dem Äquivalent von fast 2000 Jahren Rechenzeit auf einem AMD Opteron mit einem Kern von 2,2 GHz [\ldots]}\cite{kleinjung10}.
Verschlüsselungssysteme sind aufgrund dieser hohen Anforderungen gegenüber der Resistenz gegen Angriffe äußerst komplexe Systeme welche in der Informationssicherheit eine entscheidende Rolle bei der Sicherung von sensiblen Daten sowohl in der Übertragung als auch bei der Speicherung spielen\cite{aichner22}.
Das Ziel dieser Facharbeit ist es den weit verbreiteten und etablierten RSA-Algorithmus zu Analysieren und zu Implementieren. Hierfür werden die einzelnen Teilschritte und verwendeten mathematischen Prinzipien und Funktionen erläutert und beispielhaft dargestellt sowie als Teil eines Programms in C\# implementiert, so dass die Funktionen der Schlüsselerzeugung, der Verschlüsselung, der Entschlüsselung und der Signierung verfügbar sind.
\newpage

\section{Hintergrund der RSA-Verschlüsselung}

\newpage
\section{Anwendungsbereiche der RSA-Verschlüsselung}
