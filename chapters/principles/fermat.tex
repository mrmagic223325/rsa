\section{Kleiner fermatscher Satz}

Der kleine fermatsche Satz, benannt nach seinem Entdecker Pierre de Fermat, ist ein Satz in der Zahlentheorie, welcher eine Kongruenz zwischen einer natürlichen Zahl $a$ und einer Primzahl $p$ herstellt.\cite{wa01}

Wenn $a$ eine beliebige natürliche Zahl, die nicht durch $p$ teilbar ist, und $p$ eine beliebige Primzahl seien, dann gilt:

\begin{equation}
  \begin{split}
    &\left\{\,p \in \mathbb{P}\, \right\}\\
    &\left\{\,a \in \mathbb{N}\mid ggT(a,p) = 1\, \right\}\\
    &a^{p} \equiv a \pmod{p}\\
  \end{split}
\end{equation}

Dies bedeutet, dass der Rest bei der Division von $a^{p}$ durch $p$ immer $a$ ist:

\begin{equation}
  \begin{split}
    &2^{3} \equiv 2\pmod{3}\\
    &4^{17} \equiv 4\pmod{17}\\
    &5^{7} \equiv 5\pmod{7}\\
  \end{split}
\end{equation}

Es finden sich in verschiedenen Quellen\cite{wa01}\cite{mw01} auch leicht abgeänderte Formeln wie:

\begin{equation}
  \begin{split}
    &(a^{p-1}-1) \bmod p = 0\\
    &a^{p-1} \equiv 1\pmod{p}
  \end{split}
\end{equation}

Für meine Zwecke verwende ich die als drittes beschriebene $a^{p-1} \equiv 1\pmod{p}$ Formel, da die erste und zweite Formel keinen Mehrwert bei meiner Anwendung bieten.
