\subsection{Fermatscher Primzahltest}

Der kleine fermatsche Satz kann als rudimentärer Primzahltest verwendet werden, da die Kongruenz zwischen $a$ und $p$ nur gegeben ist, wenn $p$ eine Primzahl ist. Somit kann durch das iterative Testen dieser Kongruenz mit verschiedenen Basen $a$ ein Schluss auf die Primalität von $p$ gezogen werden. Die folgende Abbildung \ref{csharp:fermat} implementiert eine Funktion, welche zurückgibt ob $n$ wahrscheinlich prim oder definitiv zusammengesetzt ist. Der Parameter $k$ erlaubt es die Präzision der Funktion durch weitere Durchläufe zu verbessern.

\begin{figure}[htbp]
  \begin{xcminted}
  
  private static bool Fermat(BigInteger n, int k)
  {
    // Falls die zu überprüfende Zahl 3 oder 2 ist
    if (n == 3 || n == 2)
      // Wahr zurückgeben
      return true;
    // Führe den Primzahltest für k Iterationen aus, beziehungsweise bis er fehlschlägt
    for (int _ = 0; _ < k; _++)
    {
      // Label zu dem gesprungen wird falls keine teilerfremde Base a gefunden werden konnte
      Restart:
      // Zufällige Base a mit 2 ≤ a ≤ n-2
      BigInteger a = GetRandomBigIntInRange(n.GetByteCount(), 2, n - 2);
      // Zählt die Anzahl der Versuche eine teilerfremde Base a basierend auf der selben Zufallszahl zu finden
      int coprimeRuns = 0;
      // Label zu dem gesprungen wird falls die derzeitige Base a nicht teilerfremd ist
      NotCoprime:
      // Falls die Base a mehr als 5 mal nicht teilerfremd war wird eine neue Zufallszahl generiert
      if (coprimeRuns > 5)
        goto Restart;
      // Untersuche ob a und n teilerfremd sind
      if (BigInteger.GreatestCommonDivisor(a, n) != 1)
      {
        // erhöhe die Base a um eins
        a++;
        // erhöhe die Anzahl der Versuche um eins
        coprimeRuns++;
        // springe zum vorherigen Label
        goto NotCoprime;
      }
      // Berechne a^n-1 mod n
      BigInteger res = ModPow(a, n - 1, n);
      // Falls der Rest nicht eins ist ist n nicht prim
      if (res != 1)
      {
        // Gebe falsch zurück
        return false;
      }
    }
  }
  \end{xcminted}
  \caption{Implementation eines Primzahltestes basierend auf dem kleinen fermatschen Satz}
  \label{csharp:fermat}
\end{figure}
